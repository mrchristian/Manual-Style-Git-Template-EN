\documentclass{article}

                
\usepackage{authblk}
                
\makeatletter
                
\let\@fnsymbol\@alph
                
\makeatother
            
\usepackage{hyperref}
\author{Simon Worthington\thanks{mrchristian001@gmail.com}}
\affil{The Web}


\begin{document}

\title{System Configurations and Settings}

\maketitle

\begin{abstract}


Abstract: Lorem ipsum dolor sit amet, consectetur adipiscing elit. Nunc eu rutrum ante, tempor sagittis nunc. Etiam felis nulla, consectetur a tempor quis, luctus vel massa. Vestibulum pharetra ultricies urna at consequat. Donec volutpat enim sed lectus venenatis aliquet non ut lectus. Vivamus at orci eget arcu tempor pretium ut id.

\end{abstract}


\subsection{Introduction}\label{H9202894}



Listed here are key features of system parts with accompanying descriptions.

\begin{quote}



Block quote example: Lorem ipsum dolor sit amet, consectetur adipiscing elit. Vivamus elementum eros placerat vehicula luctus. Nulla mattis congue gravida. Aenean pulvinar, neque et tincidunt pellentesque, sapien augue sodales ipsum, eget tempor.


\end{quote}


\subsection{Fidus Writer}\label{H3154410}



\href{https://www.fiduswriter.org/}{Fidus Writer} is the collaborative editor and publishing system. Fidus Writer is licensed under the open source AGPL v.3 license. The sourcecode is available here: \href{http://github.com/fiduswriter}{github.com/fiduswriter}.


Fidus Writer functionality can be extended with plugins.


Fidus Writer has eight main areas:

\begin{enumerate}
\item Documents


\item Bibliography


\item Images


\item Templates


\item Books


\item User account


\item Administration


\item Styles - documents and books


\item Exporters


\end{enumerate}

\subsubsection{General }\label{H1068085}



PIC logged out - support, lang


Message Support - look for the speech bubble icon top right, or bottom right.


Interface language - this is set when you first use Fidus Writer and can be changed with the drop down, bottom right.


Software version number - use browser view sourcecode to read software version number. See line 10: e.g., <meta name="version" content="3.10.26">.


Keyboard shortcuts - Shift+CTRL+/


\subsubsection{1. Documents}\label{H1552973}



\subsubsection{Document view}\label{H2380763}



PIC Docs


Document saving - document are saved automatically in real-time, what you see on the screen will be saved. There is no need to use a save command.


Collaborative editing priority - Fidus Writer can be set with different priorities for which users edits or main server edit queue takes priority. The default setting is that the server takes priority, meaning it is the order in which edits arrive that takes priority rather than a designated user.


Collaborative editor visibily - if anther user is live on a document you will see their avatars top right.


Chat messaging collaborative editors - if users are live on a document you can use the chat function, bottom right. Messaging is only live when users are active. Chat is not saved and is not asynchronous.


Comments - select text and a comment icon will appear right. Click to add comment.


Spelling and grammar checking - see the \textbf{Tools} menu. Red indicated spelling, Blue grammar. Right click items to see suggested changes. Close spellcheck with \textbf{Tools > Spell/grammar check > Remove marks}. The function uses the open source software LanguageTool open source - \href{https://languagetool.org/}{https://languagetool.org/} 


Using Fidus Writer offline - TBC


Document settings - Optional sections; Citation style; Document style; Text language; Paper size, Copyright information.


Inserting Ciations - see the book icon in the toolbar. Citations can be added manually or imported from online databases, e.g., Europe PubMed Central.


Document name and document title - these are two different entities that can be linked. To start with the document title that is written into the document creates the document name. But if you want to have a different document name you need to edit the field as it appear above the document when in docuemnt editing view. Both of these parts are only edited in the document edit view.


PIC document name, document title


\subsubsection{Document manager view}\label{H7829280}



PIC doc manager


Create a document - choose a document template from the menu. A document template determines the structure of the document. Once a document template is set it cannot be changed.


Upload a Fidus Writer document - this is used primarily for uploading document revisions.


Search


Sort documents


Document selector and actions - select one or all docs with top selector. Share, copy, export, etc. 


Open documents


Access document revisions


Share documents


\subsection{Books}\label{H6665750}



\subsubsection{Automatically generated parts of a book}\label{H7619734}



Some parts of the book are generated automatically and cannot be edited directly and may depend on book settings, document templates, book styles, or specific format output filters. 


These parts are listed here.

\begin{itemize}
\item Covers


\item Table of Contents


\item Page numbering


\item Page headers and footers


\item References and footnotes: Placement as page notes or end notes, citation style


\item List of figures and tables


\item Section title pages


\item Placement of blank pages. e.g., before section titles.


\item Title page, and other front matter content divisions and styles


\end{itemize}

\subsubsection{Book setting and configurations}\label{H3989750}


\begin{itemize}
\item Basic book information


\item Book sections


\item Bibliographic citation style


\item EPUB covers


\item Printing / PDF - this is where the book layout style is set, as well as page size.


\item Validator - this check through documents for comments and track changes


\item Git - this is reserved for Publication Managers 


\end{itemize}

\subsubsection{Images}\label{H9336653}



Create categories - categories are useful for organising your images used in a publication, publication series, or withing a team or department.


Upload images - JPG, PNG, and SVG images can be uploaded.


\subsubsection{References}\label{H6668129}






\subsection{Templates}\label{H7350400}






\subsection{User account}\label{H8460509}






\subsection{Administration}\label{H4807452}






\subsection{Styles}\label{H5511626}






\subsection{Exporters}\label{H6892689}






\subsection{Contacts and sharing}\label{H8001298}



\subsubsection{How to share documents}\label{H3072165}






\subsubsection{How to share books}\label{H3372306}



Your contributors will need to have user accounts. If users do not have accounts then refer to the guide section 'What You'll Need to Get Started' for account creation instructions.


Navigate to the book site area and locate your book.


The to the right of your book click on the pencil icon and this will open up the 'sharing dialogue box.


On the left are you contacts. If you do not have any contact your will need to add them below using the add contacts icon. This will open a second dialogue box - here you will be able to invite people either via user name or email address. Once you add them you can close the dialogue box.


The users will receive an email inviting them to the project, they will also receive a notice on screen if they are logged into Fidus Writer.


Then you will need to move the users from the left to right column and give them the apppropriate access rights. For your contributors this will be as authors - make sure that the users have the pencil icon next to their name and not the eye icon, which is for viewing only.


You have now completed the process of granting users access to your publication project.


\subsection{GitLab Community Edition (GitLab CE)}\label{H7934822}








\end{document}

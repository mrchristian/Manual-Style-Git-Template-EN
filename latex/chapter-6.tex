\documentclass{article}

\usepackage{caption}
\usepackage{graphicx}
                
\usepackage{calc}
                
\newlength{\imgwidth}
                
\newcommand\scaledgraphics[2]{%
                
\settowidth{\imgwidth}{\includegraphics{#1}}%
                
\setlength{\imgwidth}{\minof{\imgwidth}{#2\textwidth}}%
                
\includegraphics[width=\imgwidth,height=\textheight,keepaspectratio]{#1}%
                
}
            
\begin{document}

\title{Schritt 3: Laden Sie Ihr Team ein}

\maketitle


\textbf{Dieser Bereich ist für }\emph{\textbf{Publikationsmanager}}\textbf{ gedacht.}


Sie können Mitwirkende zu Ihrem Publikationsprojekt einladen und ihnen Zugriff auf die Projektdokumente, sowie Büchern gewähren.


\textbf{Hinweis:} Mitwirkende können Dokumente bearbeiten und eine Vorschau der Buchveröffentlichung als PDF, E-Book und vieles mehr anzeigen, ohne die Veröffentlichung in Git zu exportieren oder andere Konfigurationen eines Buches zu ändern, z. B. um die Reihenfolge der Abschnitte (Kapitel) oder andere Buchinformationen und Einstellungen zu bearbeiten.


Wenn Ihr Team noch keine Konten besitzt, lesen Sie den Abschnitt "Was Sie für den Einstieg benötigen", um sie als Benutzer zum System hinzuzufügen.


Der Zugang zu einer Publikation ist \textbf{ein dreiteiliger Prozess} für Autoren:

\begin{enumerate}
\item Zunächst muss \textbf{der Benutzer verifizieren, dass er ein Kontakt von Ihnen ist.}


\item Danach gewähren Sie den Zugriff auf \textbf{die Bearbeitung von Dokumenten}, und


\item Zuletzt genehmigen Sie den Benutzern einen "Nur-Ansichts-Zugang“ zum Buch, \textbf{damit diese Vorschauen herunterladen können.}


\end{enumerate}

Teammitglieder können auch für verschiedene Rollen freigeschaltet werden, z. B. als Rezensenten oder Redakteure:

\begin{enumerate}
\item \textbf{Prüfer} hat die Berechtigung, Dokumente zu kommentieren, und;


\item \textbf{Redakteur} besitzt lediglich die Berechtigung, Änderungen an Dokumenten zu verfolgen.


\end{enumerate}

Die Einstellungen für diese Rollen werden am Ende des Abschnitts beschrieben.


\subsection{1. Hinzufügen von Benutzern als Kontakte}\label{H5896646}



Jeder Benutzer in Fidus Writer hat Kontakte. Ein Benutzer muss zunächst ein Kontakt sein, bevor er eingeladen werden kann, um Zugang zu Ihren Dokumenten oder Büchern erhalten zu können.


Navigieren Sie auf der Startseite von Fidus Writer zu Ihrem Benutzersymbol oben rechts und wählen Sie aus dem Dropdown-Menü "Kontakte“.

\begin{figure}
\scaledgraphics{4d32eb57-f1d6-4bf1-8730-247ee912be96.png}{1}
\caption*{Foto 1: Kontakte hinzufügen - oben rechts}\label{F39011101}
\end{figure}


Sie werden eine leere Seite sehen, wenn Sie noch keine Kontakte haben, oder eine Liste mit Kontakten.

\begin{figure}
\scaledgraphics{8587b0fe-b0f6-4603-b19b-31eca1ab07da.png}{1}
\caption*{Foto 2: Kontakt einladen - oben links. Liste der Kontakte}\label{F79849051}
\end{figure}


 


Klicken Sie oben links auf "Kontakte einladen" und fügen Sie Ihre Kontakte mittels des Benutzernamens oder der E-Mail-Adresse hinzu. Jeder hinzugefügte Kontakt wird über Ihre Kontaktanfrage benachrichtigt und muss die Anfrage annehmen.


Wenn die Person noch kein Fidus Writer-Konto hat, müssen Sie mittels E-Mail-Adresse die Person eingeladen, ein Konto zu erstellen.

\begin{figure}
\scaledgraphics{3a1c9501-2029-440a-af1e-7e7a71d59ad0.png}{1}
\caption*{Foto 3: Dialogfeld "Benutzer einladen". E-Mail-Adresse oder Benutzernamen hinzufügen, um den Benutzer einzuladen}\label{F79384891}
\end{figure}


Der Benutzer erhält über die Kontaktanfrage sowohl eine Benachrichtigung über Fidus Writer, sowie eine E-Mail, welche im Anschluss angenommen werden muss. Wenn der Benutzer bei Fidus Writer angemeldet ist, erscheint die Benachrichtigung als Pop-up-Anfrage, damit er sich zu den Kontakten durchklicken kann. Außerdem können Sie jederzeit ihre Kontaktbereiche besuchen, um Ihre Anfrage zu überprüfen.


Sie können den Status einer Einladung für einen Kontakt in Ihrem Kontaktbereich einsehen. Der Status einer Einladung ändert sich zu "Benutzer" sobald der Kontakt die Anfrage angenommen hat.

\begin{figure}
\scaledgraphics{bea36502-ae28-444b-a70f-7fd8917ed085.png}{1}
\caption*{Foto 4: Der Status einer Einladung ist Notizen als Benutzer, wenn die Einladung angenommen wurde.}\label{F51031911}
\end{figure}


Wenn Sie Probleme beim Hinzufügen von Kontakten haben, wenden Sie sich an den Verwaltungssupport, der Ihnen helfen kann, den Status der Einladungen zu überprüfen. Alle personenbezogenen Daten werden unter strikter Einhaltung der DSGVO und der Grundsätze der digitalen Souveränität verwendet, wobei die Nutzer immer ausdrücklich Zugang zu ihren personenbezogenen Daten gewähren müssen.


\subsection{2. Benutzern den Zugang zur Bearbeitung von Dokumenten gewähren}\label{H2959514}



\textbf{Hinweis:} Als Ersteller von Dokumenten werden Sie automatisch zum Dokumenteneigentümer. Jedes Dokument kann lediglich einen Dokumenteneigentümer besitzen. Nur der Eigentümer eines Dokuments kann die Freigabeeinstellungen bearbeiten. Benutzer, die Sie einladen, können alle Teile eines Dokuments bearbeiten. Dies beinhaltet zudem das Löschen von Dokumenten, da wir Ihnen mit der Freigabe den Schreibzugriff zu allen Dokumente gewähren. Sie können den Zugriff auch folgendermaßen einstellen: Verfolgtes Schreiben (Änderungen nachverfolgen); Kommentar, oder; Lesen (nur lesen).


Navigieren Sie zur Startseite von Fidus Writer und zum Bereich „Dokumente“ und von dort aus in das Verzeichnis, das Sie im vorherigen Schritt der Anleitung erstellt haben. Hier sehen Sie eine Liste Ihrer Publikationsdokumente.

\begin{figure}
\scaledgraphics{257455fd-bc94-461a-a77e-7b0bc27dc03d.png}{1}
\caption*{Foto 5: Publikationsdokumente}\label{F62151421}
\end{figure}


Aktivieren Sie im Verzeichnis die oberen Kontrollkästchen über allen Dokumenten, um die Auswahl aller Dokumente ein- und auszuschalten, klicken Sie dann auf das Dropdown-Pfeilsymbol und wählen Sie "Teilen" aus dem Dropdown-Menü.

\begin{figure}
\scaledgraphics{4aaf5a28-7970-4f78-bb98-e1232313f54b.png}{1}
\caption*{Foto 6: Wählen Sie alle Dokumente aus, indem Sie das Kontrollkästchen über den Dokumenten aktivieren. Beachten Sie, dass das Dropdown-Menü für die Freigabe aus dem Pfeil nach unten rechts neben dem Kontrollkästchen besteht.}\label{F49894591}
\end{figure}


Sie sehen nun das Dialogfeld für die Freigabe. Fügen Sie Benutzer hinzu, indem Sie sie von der linken in die rechte Spalte verschieben, und ändern Sie das Symbol neben jedem Benutzer von "Anzeigen" (Augensymbol) in "Bearbeiten" (Bleistiftsymbol) um, um Ihnen vollen Bearbeitungszugriff zu gewähren, anderenfalls können Benutzer:innen die Dokumente nur anzeigen. Speichern Sie im Anschluss Ihre Freigabeeinstellungen.

\begin{figure}
\scaledgraphics{2b3de32d-4d3e-4e7c-9161-6d7d37b1c232.png}{1}
\caption*{Foto 7: Wählen Sie den Menüpunkt Freigeben aus dem Dropdown-Dokumentenmenü}\label{F11743441}
\end{figure}


Die Freigabe von Dokumenten ist nun abgeschlossen.


\textbf{Wenn Sie einen neuen Benutzer oder ein neues Dokument hinzufügen, wiederholen Sie die Teile 1. und 2. um die Freigabe zu ermöglichen.}


\subsection{3. Freigabe des Buchs zur Ansicht und zum Download der Vorschau}\label{H5996080}



Sie möchten, dass Ihre Mitwirkenden die Möglichkeit haben, die Bucheinstellungen einzusehen und eine Vorschau des gesamten Buches in seinen verschiedenen Satzlayoutformaten zu sehen, aber Benutzer:innen sollen nicht in der Lage sein, das Buch zu veröffentlichen, direkt Buchabschnitte neu anzuordnen oder einen neuen Satzstill für das Layout auszuwählen.


In diesem Teil werden wir das Buch für Benutzer:innen freigeben wie zuvor, mit dem Unterschied, dass diese nun nur berechtigt sind, Dokumente "\textbf{Nur anzeigen}" zu lassen.


1. Navigieren Sie hierfür zum Bereich "Bücher“ von Fidus Writer und suchen Sie Ihr Buch.

\begin{figure}
\scaledgraphics{a2dd5291-f50f-43c5-a251-a2696cf99f7c.png}{1}
\caption*{Foto 8: Bereich der Buchseite}\label{F10022161}
\end{figure}


2. Klicken Sie rechts neben Ihrem Buch auf das Bleistiftsymbol. Daraufhin wird das Dialogfeld für die gemeinsame Nutzung angezeigt. Wie bei der Freigabe von Dokumenten verschieben Sie die Benutzer von der linken Spalte in die rechte Spalte, um das Dokument mit Ihnen zu teilen. Der Unterschied besteht diesmal darin, dass wir die Benutzer \textbf{als reine Betrachter (Augensymbol)} berechtigen.


Sobald Sie diesen Teil abgeschlossen haben, ist die Einrichtung der Freigabe insgesamt abgeschlossen.

\begin{figure}
\scaledgraphics{307a65dd-0aa1-43ba-97aa-0811ec5d396a.png}{1}
\caption*{Foto 9: Teilen Sie Ihre Veröffentlichung. Bearbeiten Sie das Bleistiftsymbol rechts neben dem Buch. Fügen Sie dann im Dialogfeld des Buches in der rechten Spalte weitere Benutzer:innen hinzu und setzen Sie sie auf Bearbeiten (Bleistiftsymbol).}\label{F50686771}
\end{figure}


\subsection{Hinzufügen von Prüfern und Bearbeitern zu Dokumenten}\label{H5555602}



\subsection{Hinzufügen von Prüfern und Bearbeitern zu Dokumenten}\label{H4871756}



Für Dokumente besitzen Sie die Möglichkeit, die Zugriffsrechte eines Benutzers auf "Nur anzeigen", "Nur kommentieren" oder "Nur Änderungen Nachverfolgen" festzulegen.


Diese Einstellungen sind für Prüfer und Redakteure nützlich.

\begin{figure}
\scaledgraphics{ab818b4d-2c97-46c1-94eb-b3c6f50a112a.png}{1}
\caption*{Foto 10: Optionen für die Freigabe von Dokumenten für Mitwirkende und Überprüfer (Basis: Schreiben, Schreiben mit Spuränderung, Kommentar, Lesen. Rezension: keine Kommentare, Rezension, Reviewer mit nachverfolgten Änderungen)}\label{F40650491}
\end{figure}


\subsubsection{Basic (Mitwirkende)}\label{H2272325}


\begin{itemize}
\item Schreiben


\item Schreiben mit Spuränderung


\item Kommentar


\item Lesen


\end{itemize}

\subsubsection{Rezension}\label{H6146355}


\begin{itemize}
\item Keine Kommentare


\item Rezension


\item Reviewer mit nachverfolgten Änderungen


\end{itemize}

\subsection{Nächste Schritte}\label{H8503735}



Als Nächstes werden wir uns mit der Ausgabe Ihrer Publikation in Git befassen. Dies ist der vierte und letzte Schritt in diesem Leitfaden für Ihren Publikations-Arbeitsablauf.

\end{document}

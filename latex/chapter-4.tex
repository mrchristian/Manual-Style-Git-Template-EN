\documentclass{article}

\begin{document}

\title{Step 3: Invite Your Team}

\maketitle


You can invite contributors to your publication project and give them access to project documents and book.


\textbf{Note:} This setup is so that contributors can edit documents and preview the book publication as PDF, e-book, etc., without being able to export the publication to Git, change the order of sections (chapters), or edit other book information and settings.


If you do not have an account yet as a contributor or as a publication manager you need to have your team added as users then see account creation in the guide section 'What You'll Need to Get Started'.


For contributors access to a publication is a three part process:

\begin{enumerate}
\item First, the user has to accept being a contact of yours.


\item Second, you grant document editing access, and 


\item Third, you can give view-only access to the book so that users can download previews.  


\end{enumerate}

Team members can also be give access for different roles, these roles are described at the end of the section, as:

\begin{enumerate}
\item reviewer with commenting only on documents; 


\item as an editor with track changes permissions on documents;


\end{enumerate}

\subsection{1. Adding users as Contacts}\label{H3777938}



In the homepage of Fidus Writer navigate to your user icon top right and from the drop-down menu select Contacts.


PIC


You will see an empty page if you have no contacts yet or a list of contacts.


PIC


Click add contacts top left, you can add contacts here by username or email address. Each contact added will be notified about your contact request and will need to approve the request.


PIC


The user will get a notice in Fidus Writer and as email about the contact request, and then they need to accept the request. The notice will come up as a pop-up request for them to click through to contacts. Also, they can always visit their contact areas to check on your request.


PIC 1


PIC 2


You can see the status of your invite for a contact in your contacts view area.


PIC


If you have problem adding contact then get in touch with administration support, and they can help check on the status of invites etc. All personal information is used in strict adherence to GDPR and principles of Digital Sovereignty where users always have to grant explicit access to their personal data.


\subsection{2. Giving users access to edit documents}\label{H9350209}



Navigate to the Fidus Writer home and the documents area and from there into the directory you made in the earlier step in the guide. Here you will see a list of your publication documents.


PIC Docs top level


PIC list of documents in directory


In the directory select the top checkboxes above all the document checkboxes this will turn on and off (toggle) the selection of all the documents, then click the drop-down icon and select 'Share' from the drop-menu.


PIC share


You will now see the share dialogue box. Add users by moving them from the left to the right column and edit icon next to each user and change it from the view (eye icon) to edit (pencil icon) to give them full edit access, otherwise they will only be able to view documents. And then save your sharing settings. 


The sharing task for documents is now complete.


If you add a new user or new document - then repeat parts 1. and 2. again.


\subsection{3. Sharing your book for view only and preview download}\label{H6251349}



You want your contributors to be able to view the book settings and preview the complete book in its different typeset layout formats, but prevent them from publishing the book or directly rearranging book sections or selecting a new layout typesetting style, etc.


In this part we will share the book with the same users as before in documents, but with the permissions as view only.


1. Navigate to the book site area of Fidus Writer and locate your book.


PIC


2. To the right of your book click the pencil icon. This will bring up the sharing dialogue box. As before with sharing documents move the users from the left column to the right column to share the document with them. The difference this time is that we're going to leave the users as view only (eye icon).


Once you have completed this part the sharing setup is completed.


PIC view


\subsection{Adding reviewer and editors to documents}\label{H5018454}



For documents, you have the option to set a users access rights as view only, comment only, or as track changes only.


These setting are useful for reviewers and editors.


PIC showing drop-down


\subsection{Next steps}\label{H4422568}



Next we will look at outputting your publication to Git. This will be the fourth and final step in this guide for your publication workflow.

\end{document}

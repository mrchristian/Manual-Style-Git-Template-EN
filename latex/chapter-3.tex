\documentclass{article}

\begin{document}

\title{Pipeline Vorteile}

\maketitle


Single-Source-Publishing: Bearbeiten Sie die Dateien an einem Ort und verteilen Sie sie automatisch als Multiformat an verschiedene Standorte, um professionelle, publikationsfertige Ausgaben zu erstellen.

\begin{itemize}
\item Kein Austausch von Dateien per E-Mail mehr


\item Keine Verwirrung mehr darüber, welches die richtige Dokumentversion ist


\item Keine Verzögerungen mehr beim Warten auf Bearbeitungen und Überprüfungen


\item Keine langwierigen Wartezeiten mehr für die Rücksendung von Layoutentwürfen


\item Entfall der Komplexität der Veröffentlichung in mehreren Formaten und das Nachverfolgen von Korrekturen und Änderungen


\item Keine Verzögerungen mehr bei der Verteilung an interne oder externe Kanäle


\end{itemize}

\textbf{Veröffentlichen Sie in mehreren Formaten:} PDF; Print-on-Demand (PoD); Web (mobile first); seitenbasiertes Web; E-Book; interoperable Formate - JATS, DOCX, HTML, EPUB, LaTeX, JSON, etc.


\textbf{Sofortiger automatischer Schriftsatz und Layout-Design:} Wiederverwendbare Layout-Vorlagen für den Satz von Multiformat-Ausgaben auf Knopfdruck. Dies bedeutet keine zeitraubende Verzögerung bei der Layoutgestaltung während der Produktion. Stattdessen kann der Layouter die Designvorlagen im Voraus erstellen.


\textbf{Vorschau von Multiformat-Layouts:} Mitwirkende an der Publikation haben die Möglichkeit eine Vorschau der verschiedenen Multiformat-Ausgaben zu erhalten.


\textbf{Ko-Kreation:} Es wird ein Online-Echtzeit-Textverarbeitungsprogramm verwendet, das es mehreren Benutzern ermöglicht, gleichzeitig zusammen Dokumenten zu bearbeiten. Das bedeutet, dass Autoren, Redakteure, Prüfer und Designer gleichzeitig an einem Dokument arbeiten können.


\textbf{Hochwertiges akademisches Textverarbeitungsprogramm:} Zu den Funktionen gehören der Zugriff auf Zitierdatenbanken wie EuropePMC, Fußnoten und Zitate, Zitierformate, optionale Abbildungen und Tabellenbeschriftungen sowie Listen.


\textbf{Versionierung:} Es wird ein Versionsspeicher verwendet, in dem Ausgaben veröffentlicht werden können, wobei alle früheren Versionen verfügbar sind. Außerdem werden alle Bearbeitungen gespeichert, so dass Änderungen nachverfolgt, geprüft und bei Bedarf rückgängig gemacht werden können. Die Versionierung wird mit kryptografischen IDs aufgezeichnet, um eine präzise Bearbeitung und Validierung der Dokumentversionen zu ermöglichen.


\textbf{Automatische Verteilung:} Multiformat-Ausgaben können an interne Organisationen oder externe Standorte und Kanäle verteilt werden.


\textbf{Automatische Erstellung von Websites:} Für Veröffentlichungen können automatisch Websites erstellt werden, die öffentlich oder privat sein können.


\textbf{GitLab-Infrastruktur:} Die GitLab Community Edition wird als selbstgehostete Option für die Speicherung von Publikationen verwendet. GitLab bietet eine leistungsstarke Infrastruktur für die Verarbeitung und Verteilung von Inhalten, automatisierte Aufgaben und Teamarbeit.


\textbf{Semantisch durchdacht:} Von Beginn der Bearbeitung an sind die Publikationsinhalte semantisch strukturiert - dies ist der Schlüssel zur Automatisierung. Zusätzliche Ebenen der semantischen Strukturierung können hinzugefügt werden, um Dokumentenstrukturen Bedeutung zu verleihen. Zusätzlich können Linked Open Data, Ontologien und kontrollierte Vokabulare verwendet werden, um die Bedeutung der Publikationsinhalte zu strukturieren.


\textbf{Verbesserte Publikationen:} Das moderne Open-Science-Publishing bietet eine Reihe von Erweiterungen der Publikationsfunktionen, die eingesetzt werden können. Persistente Identifikatoren (PIDs) zur korrekten Identifizierung von Organisationen (ROR), Personen (ORCID) und Dokumenten (DOI). Die offene Lizenzierung gewährleistet bei Bedarf die Wiederverwendung. Interoperable Formate gewährleisten Wiederverwendung, Auffindbarkeit und Portabilität. Maschinenlesbare Inhalte und Metadaten gewährleisten, dass die Inhalte FAIR-konform sind (Findability, Accessibility, Interoperability, Reusable). Linked Open Data Markup werden verwendet, um sicherzustellen, dass die Inhalte strukturiert und in Wissenssystemen und KI/ML wiederverwendbar sind.


\textbf{Digitale Souveränität durch Design:} Bei allen Aspekten des Systems werden personenbezogene Daten, der Schutz der Privatsphäre und die Datensicherheit berücksichtigt. Zu diesen Maßnahmen gehören die Einhaltung der GDPR, die Verwendung von Open-Source-Software für Code-Audits, die Verwendung von sicherem Self-Hosting statt Vor-Ort-Hosting bis hin zum Cloud-Hosting mit ausgewiesener Zuständigkeit sowie sichere DevOps-Verfahren.

\end{document}

\documentclass{article}

\usepackage{caption}
\usepackage{wrapfig}
\usepackage{svg}
\usepackage{graphicx}
                
\usepackage{calc}
                
\newlength{\imgwidth}
                
\newcommand\scaledgraphics[2]{%
                
\settowidth{\imgwidth}{\includegraphics{#1}}%
                
\setlength{\imgwidth}{\minof{\imgwidth}{#2\textwidth}}%
                
\includegraphics[width=\imgwidth,height=\textheight,keepaspectratio]{#1}%
                
}
            
\begin{document}

\title{Willkommen in der Publishing Pipeline!}

\maketitle

\begin{figure}
\scaledgraphics{b3bc6ba5-db38-4ec8-89db-45c029fdb485.png}{0.75}
\label{F98934631}
\end{figure}


In der Schnellstartanleitung erfahren Sie, wie Sie die "Publishing-Pipeline" für die Erstellung von Publikationen in mehreren Formaten nutzen können: Berichte, Handbücher, Bücher, Abhandlungen und vieles mehr.


Die "Publishing Pipeline" verbindet das Textverarbeitungsprogramm mit dem Veröffentlichen. Für die Publikation bedeutet dies, dass Sie von einem Online-Mehrbenutzer-Editor aus automatisch Ausgaben in mehreren Formaten erstellen und umwandeln können - PDF, Web, eBook, Print-on-Demand und mehr, welche für die Dateispeicherung oder online verfügbar sind. Außerdem können Sie jederzeit Änderungen an allen Formatausgaben aus einer einzigen Quelle vornehmen.


Hochwertige Layout-Designs werden durch die Kombination von vorgefertigten Layout-Designstilen mit automatisiertem Maschinensatz ermöglicht. Dies bedeutet, dass alle zeitaufwändigen Layout-Design-Arbeiten aus der Projektzeitraum herausgenommen und im Voraus erledigt werden, was einen schnellen Arbeitsablauf bei der Veröffentlichung ermöglicht.


Die Schnellstartanleitung richtet sich an Autoren und Publikationsmanager. Technische Administratoren und Entwickler sowie Schriftsetzer sollten sich den 'Admin Guide' ansehen.


Wir werden mit einem kollaborativen Online-Textverarbeitungsprogramm arbeiten und in verschiedenen Formaten veröffentlichen - PDF, Web, E-Book, Mobile, Print-on-Demand und vieles mehr. - und dabei eine digital souveräne Software und Systeme verwenden, um den Datenschutz und die Sicherheit zu gewährleisten, dies beinhaltet Selbst veranstalten, GDPR-Konformität, Verschlüsselung und so weiter.


\subsection{Was Sie brauchen, bevor Sie beginnen}\label{H4632171}


\begin{wrapfigure}{r}{0.25\textwidth}
\scaledgraphics{fec439eb-c4d6-4587-a4a2-affde7c45586.png}{0.25}
\label{F18710661}
\end{wrapfigure}


Im Abschnitt "Was Sie für die ersten Schritte benötigen" dieses Leitfadens finden Sie Anweisungen zum Erstellen aller erforderlichen Konten.


\subsubsection{Für Nutzer:innen}\label{H6462936}



Die Teilnehmer müssen Folgendes mitbringen.

\begin{enumerate}
\item Eine E-Mail-Adresse für den Empfang von Account-E-Mails.


\item Ein Benutzerkonto für das Online-Textverarbeitungsprogramm "Fidus Writer".


\end{enumerate}

\subsubsection{Für Publikationsleiter:innen}\label{H5586293}



Die Verantwortlichen für Publikationen benötigen Folgendes.

\begin{enumerate}
\item Eine E-Mail-Adresse für den Empfang von Konto-E-Mails.


\item Ein Benutzerkonto für das Online-Textverarbeitungsprogramm "Fidus Writer".


\item GitLab- oder/und GitHub-Konten, je nach unterstützter Git-Plattform Sie verwenden.


\item Verbinden Sie "Fidus Writer" mit der Git-Plattform Ihrer Wahl.


\end{enumerate}

\subsection{Die Schritte zur Erstellung einer Publikation}\label{H5940338}


\begin{enumerate}
\item Erstellen eines Git-Repositorien und einer Website


\item Ein Buch erstellen (Zusammenstellung von Dokumenten)


\item Das Team einladen


\item Wie man im Multiformat veröffentlicht


\end{enumerate}

\subsection{Was Sie hier lernen werden}\label{H7757657}


\begin{enumerate}
\item Einrichtung eines Kontos für Fidus Writer, GitLab, einschließlich GitLab.com und GitLab CE, und GitHub.


\item Wie Sie Ihr öffentliches Git-Repository für die Speicherung Ihrer Publikationsdaten vorbereiten, mit einer Option zur Aktivierung einer Website.


\item Erstellung von GitLab Pages und GitHub Pages-Websites.


\item Wie Sie die kollaborative Online-Textverarbeitung für Ihre Publikation einrichten.


\item Laden Sie Ihr Team ein, um online kollektiv an Dokumenten zu arbeiten.


\item Wie man veröffentlicht.


\end{enumerate}

\subsection{Pipeline-Merkmale}\label{H2087393}


\begin{itemize}
\item Gemeinsamer Arbeitsbereich: Laden Sie Designer, Redakteure, Korrekturleser oder Lektoren ein, an einer Publikation zu arbeiten.


\item Erstellung von Publikationen in mehreren Formaten: Website, PDF, paginiertes Web, eBook, Print-on-Demand usw.


\item Automatischer Satz und Layout-Designstile, so dass kein zeitaufwändiger Schriftsatz erforderlich ist.


\item Single-Source-Publishing: Bearbeitung und Verteilung an alle Formate.


\item Zitier-Manager.


\item Open-Source-Software und "Pipeline-Architektur" für die Systemintegration.


\item Git-Speicher mit Versionierung.


\item Interoperable Formate: JATS/XML, JSON, HTML, LaTeX, usw.


\item Semantische Strukturierung und Anreicherung: Linked Open Data (Verwendung von Terminologiediensten und TDM), PID auf Publikationsebene, interne Struktur der Publikation und für digitale Objekte.


\end{itemize}

\subsection{Systemkonfigurationen und Einstellungen}\label{H1641086}



Informationen zu den Einstellungen von Fidus Writer, Dokumenten und Büchern finden Sie im Abschnitt "Systemkonfigurationen und -einstellungen" des Handbuchs.


\subsection{Struktur der Veröffentlichungsdaten}\label{H4187554}


\begin{figure}
\includesvg[width=0.75\textwidth]{cc9145cb-1270-411e-b526-66011ded1c49.svg}
\caption*{Abbildung 1: Datenmodell des Systems}\label{F86281041}
\end{figure}


\subsection{Digitale Souveränität}\label{H6425485}


\begin{wrapfigure}{l}{0.5\textwidth}
\scaledgraphics{60a0cd01-c4e1-467e-a517-a5ae77dbbbaf.png}{0.5}
\label{F43863211}
\end{wrapfigure}


Der Begriff "digitale Souveränität" wird hier verwendet, um die Schritte zu beschreiben, die unternommen werden, um die Privatsphäre persönlicher Daten und die Sicherheit von Inhalten zu gewährleisten. Datenschutz und Sicherheit sind von entscheidender Bedeutung, da Unternehmen, Staaten, beziehungsweise Parteien mit böswilligen Absichten oder durch versehentliche Datenverluste in die digitalen Aktivitäten eingreifen können.


Um Ihre "digitale Souveränität" zu gewährleisten, kombinieren wir Datensicherheitsmaßnahmen, die Einhaltung von Datenschutzgesetzen, wie der Europäischen Datenschutz-Grundverordnung (GDPR) und die Bereitschaft zur Einhaltung von Datenschutzgesetzen verschiedener Zuständigkeiten wie dem California Consumer Privacy Act (CCPA. Zudem gewährleistet die Publishing-Pipeline die Transparenz von Code und Datenspeicherung.


Das System kann selbst gehostet werden, ist Open-Source und erfüllt die Anforderungen der DSGVO. Es verwendet die Zwei-Faktor-Authentifizierung für Verwaltungsbereiche und die OAuth-Authentifizierung für die Integration der Authentifizierungs- und Autorisierungsinfrastruktur (AAI).

\end{document}

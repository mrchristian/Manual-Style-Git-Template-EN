\documentclass{article}

\usepackage{caption}
\usepackage{graphicx}
                
\usepackage{calc}
                
\newlength{\imgwidth}
                
\newcommand\scaledgraphics[2]{%
                
\settowidth{\imgwidth}{\includegraphics{#1}}%
                
\setlength{\imgwidth}{\minof{\imgwidth}{#2\textwidth}}%
                
\includegraphics[width=\imgwidth,height=\textheight,keepaspectratio]{#1}%
                
}
            
\begin{document}

\title{Schritt 2: Erstellen eines Buchprojekts in Fidus Writer}

\maketitle


Das Buchprojekt in Fidus Writer dient als leerer Ablageort für Ihre Publikation. Später können Sie alle Dateinamen und Buchinformationen ändern, um Titel und Inhalt Ihres Buches wiederzugeben. Zusätzlich können Sie auch jederzeit Dokumente hinzufügen und entfernen.


\subsection{Was hier behandelt wird}\label{H9231515}


\begin{enumerate}
\item Erstellen Sie einen "persönlichen" Ordner (den nur Sie sehen - er ist nicht freigegeben) für Ihre Buchdokumente


\item Erstellen Sie drei Platzhalterdokumente für Ihre Buchteile: „Vorderseite“, „Abschnitt 1“, und „Rückseite“


\item Fügen Sie Ihre Dokumente zu einem Fidus Writer Book hinzu - hierbei handelt es sich um eine Zusammenstellung von Buchdokumenten


\item Verbinden Sie Ihr Buch mit einem Git Repository


\end{enumerate}

In einem späteren Schritt wird die gemeinsame Nutzung der Publikation mit Ihrem Team behandelt.


Vollständige Details zur Konfiguration der Publikation finden Sie im Pipeline-Handbuch.


\subsection{1. Erstellen Sie einen "persönlichen" Ordner}\label{H4616269}



Hier erstellen Sie einen Ordner und legen anschließend Ihre drei Dokumente in diesem Ordner ab. Zu Beginn müssen Sie sich im Dokumentenbereich der Website befinden.


Klicken Sie oben auf der Seite im Sekundärmenü auf "Neuen Ordner erstellen" und geben Sie dem Ordner einen Namen.

\begin{figure}
\scaledgraphics{04d64821-e590-4586-bdb7-8057ed52a4a0.png}{1}
\caption*{Foto 1: Erstellung von Dokumentenordnern - Ordner hinzufügen, benennen und speichern}\label{F97885011}
\end{figure}


Jetzt haben Sie einen leeren Ordner. Wenn in dem Ordner keine Dokumente erstellt werden und er leer bleibt, wird der Ordner nicht gespeichert, wenn Sie aus dem Ordner rausgehen.


\subsection{2. Platzhalterdokumente erstellen}\label{H4781105}



Wir werden nun drei Dokumente in dem soeben angelegten Ordner erstellen. Dies sind die Beispiele für die Platzhalterdokumente, die Sie einrichten werden:

\begin{itemize}
\item Titelblatt: Hier fügen Sie das Impressum, Informationen zu den Mitwirkenden, Danksagungen usw. ein.


\item Abschnitt 1: Ein übergeordneter Teil eines Buches als Abschnitt oder Kapitel


\item Rückseite: Dieser Teil kann Anhänge, Glossare, Abkürzungen usw. enthalten.


\end{itemize}

\subsubsection{So erstellen Sie Dokumente}\label{H5646695}



Wählen Sie im Untermenü unter Dokumente die Option \textbf{"Neues Dokument erstellen"} und wählen Sie die Dokumentvorlage "Buchstandard". Wenn Sie an einem speziellen Buch oder einer Publikationsreihe arbeiten, können Sie eine andere Dokumentvorlage verwenden. Wenden Sie sich hierfür bezüglich einer Beratung an ihren Publikationsmanager.


Hier werden Sie drei Dokumente als Platzhalter hinzufügen. Diese werden hinzugefügt, damit Sie Ihre Buchgrundlagen konfigurieren können. Hier sollte erwähnt werden, dass Namen und Dokumente später geändert oder gelöscht werden können. Legen Sie drei Dokumente mit den folgenden Namen an: Vorderseite; Abschnitt 1, und; Rückseite.

\begin{figure}
\scaledgraphics{252b11a1-2b67-4b4a-8445-2a5277dab934.png}{1}
\caption*{Foto 2: Dokumente erstellen und eine Dokumentvorlage verwenden}\label{F79921751}
\end{figure}


1. Neues Dokument erstellen, 2. die Dokumentvorlage auswählen, 3. den Titel des Dokuments hinzufügen, 4. das Dokument über das Menü Datei schließen.

\begin{figure}
\scaledgraphics{73c56889-a204-4b6b-829f-f064a973bccd.png}{1}
\caption*{Foto 3: Erstellen Sie ein Dokument und fügen Sie einen Titel hinzu, dann schließen Sie es über das Menü Datei}\label{F9841231}
\end{figure}

\begin{figure}
\scaledgraphics{58dc095f-4699-43ab-b89b-d4b6898c429e.png}{1}
\caption*{Foto 4: Hinzufügen von Dokumenten, die in Ihrem Buch verwendet werden sollen}\label{F19682601}
\end{figure}


Sie haben nun die grundlegenden Buchabschnitte, und wir können mit der Erstellung des Fidus Writer Buchablageorts fortfahren.


\subsection{3. Ein Fidus Writer Buch erstellen}\label{H853693}



Ein Fidus Writer Buch fasst eine Reihe von Fidus Writer Dokumenten zusammen. Hier werden wir ein Buch erstellen und Ihre soeben erstellten Dokumente hinzufügen, sowie einige grundlegende Konfigurationen des Buches vornehmen.


Navigieren Sie zum Abschnitt "Bücher“ auf der Website.

\begin{figure}
\scaledgraphics{8b772d23-b634-4ed7-99cb-592991930b27.png}{1}
\caption*{Foto 5: Erstellen Sie ein Fidus Writer-Buch. Navigieren Sie zum Abschnitt "Buch" und verwenden Sie "Buch erstellen" auf der linken Seite}\label{F33363991}
\end{figure}


Klicken Sie auf "Neues Buch erstellen". Es wird ein Buchdialogfeld mit einer Reihe von Registerkarten angezeigt: Basisangaben, Kapitel, Bibliografie, Epub, Druck/PDF, Validierung und Git repository.


Zu Beginn werden Sie nur einige wenige Einstellungen vornehmen. Sie können später zurückkehren, um die gesamte Konfiguration des Buches abzuschließen. Hier werden wir den Titel ausfüllen und Ihre Dokumente hinzufügen.


1. Geben Sie den Buchtitel auf der Registerkarte „Basisangaben“ ein.

\begin{figure}
\scaledgraphics{ba321105-5867-457f-a1d8-5722ae3ffc7e.png}{1}
\caption*{Foto 6: Buchinformationen hinzufügen, Buchtitel hinzufügen, um damit zu beginnen}\label{F70457681}
\end{figure}


2. Dokumente hinzufügen. Um Ihre Dokumente hinzuzufügen, wechseln Sie zur Registerkarte "Kapitel". Hier sehen Sie auf der linken Seite Ihre Dokumente aufgelistet, ganz oben Ihren neu erstellten "Ordner". Klicken Sie auf den Ordner, um seinen Inhalt anzuzeigen. Sie können Ihre Dokumente dem Buch hinzufügen, indem Sie diese auswählen und auf den Pfeil in der Mitte klicken, um sie der rechten Spalte hinzuzufügen. Speichern Sie nun Ihr Buch. Das Dialogfeld wird nun geschlossen, und Ihr Buch wird in der Rubrik "Bücher" der Website aufgeführt.

\begin{figure}
\scaledgraphics{e3b528b3-71a2-4efe-990d-e9914c328695.png}{1}
\caption*{Foto 7: Wählen Sie die Registerkarte "Kapitel" und fügen Sie Dokumente aus Ihrem Ordner in die rechte Spalte ein, um sie in Ihr Buch aufzunehmen. Dann speichern}\label{F19256461}
\end{figure}


Sie können später zurückkehren, um alle Bucheinstellungen zu vervollständigen.


Ihr Buch ist nun bereit, mit Git verbunden zu werden, um es zu veröffentlichen.


\subsection{4. Verbinden Sie Ihr Fidus Buch mit einem Git Repository}\label{H4837045}



Dieser Teil des Prozesses muss nur von Publikationsmanagern oder Benutzern durchgeführt werden, welche für die Ausgabe mit Git verantwortlich sind. Wenn das Git-Repository öffentlich ist, kann jeder Benutzer die gespeicherten Inhalte ohne Anmeldedaten einsehen. Repos können privat eingestellt werden oder der Zugriff kann nur bestimmten Benutzern oder Benutzergruppen gewährt werden.


Voraussetzung für den nächsten Schritt ist es, dass Sie ihr Git-Repository bereits erstellt haben, wie in Schritt 1 des Leitfadens beschrieben, denn in diesem Repository werden Sie auch Ihre Publikationsdateien ablegen.


Zuerst verbinden wir Fidus Writer mit der von Ihnen verwendeten Git-Instanz, indem wir Git autorisieren, sich mit Fidus Writer zu verbinden, indem wir Ihre Benutzerkonten auf beiden Systemen verwenden.


\subsubsection{Plattformen verbinden}\label{H4178971}



1. Stellen Sie sicher, dass Sie bei Git und Fidus Writer angemeldet sind.


2. Navigieren Sie auf der Fidus Writer-Startseite oben rechts hin zu Ihrem Benutzerprofil und klicken Sie auf Ihren Benutzernamen, um zu Ihrer Benutzerprofilseite zu gelangen, wo Sie sich im Bereich "Soziale Konten" mit Ihrer Git-Instanz verbinden können.


3. Klicken Sie neben der Git-Instanz, mit der Sie sich verbinden möchten auf „Verbinden“.

\begin{figure}
\scaledgraphics{b9ca1acc-be74-4f8b-9603-368cc2bf5f3a.png}{1}
\caption*{Foto 8: Mit Git verbinden}\label{F94991061}
\end{figure}


4. Sie werden nun auf die Git-Website weitergeleitet und müssen sich anmelden, falls Sie dies noch nicht getan haben.


5. Akzeptieren Sie dann die Autorisierung. Durch diesen Vorgang werden Ihre Benutzerkonten verbunden und die beiden Systeme können Ihre Publikationsdateien übertragen.


Der Verbindungsprozess ist nun abgeschlossen, und wir werden nun das Repository für Ihr Buch auswählen.


\subsubsection{Repository auswählen}\label{H3752715}



1. Navigieren Sie zu Ihrem Buch und klicken Sie darauf, um das Buchdialogfeld zu öffnen. Klicken Sie auf die Registerkarte Git-Repository auf der rechten Seite.

\begin{figure}
\scaledgraphics{03dc5c4f-e517-4c5e-89e2-c58d03d72a7b.png}{1}
\caption*{Foto 9: Wählen Sie das zu verwendende Repository und speichern Sie es. Neu laden, falls das gewünschte Repository nicht verfügbar ist.}\label{F97152971}
\end{figure}


2. Klicken Sie auf "Aktualisieren" auf der rechten Seite, um die Liste der Repositories von Git zu erhalten. Die Repositories werden nun im Dropdown-Menü verfügbar sein.


3. Wählen Sie Ihr Repository aus der Liste aus, markieren Sie unten die gewünschten Ausgabetypen und klicken Sie auf "Speichern". Die Optionen für das Exportformat sind: EPUB-Export, Ungepackter EPUB-Export, HTML-Export, Export Unified HTML, LaTeX-Export. Standardmäßig benötigen Sie nur EPUB und Unified HTML. PDF wird manuell hochgeladen - Anweisungen dazu finden Sie im Abschnitt "Schritt 4: Als Multiformat veröffentlichen!".


4. Sie können nun Ihr Buch nach Git exportieren. Klicken Sie auf der Registerkarte Git-Repository unten rechts auf die Schaltfläche Export und wählen Sie "Export to Git repository". Es erscheint ein Dialog, in dem Sie aufgefordert werden, eine Commit-Nachricht einzugeben, die eine Notiz für den Revisionsexport darstellt.

\begin{figure}
\scaledgraphics{e8845c51-1622-4ec7-aa72-ee8b7e6abe4f.png}{1}
\caption*{Foto 10: Git-Export-Einstellungen. Registerkarte "Git"; Repository auswählen; Ausgaben wählen und exportieren}\label{F11208021}
\end{figure}


Unten rechts erscheint ein Meldungsdialog. Wenn die Meldung "Buch erfolgreich im Repository veröffentlicht!" erscheint, ist der Vorgang abgeschlossen.

\begin{figure}
\scaledgraphics{de2690fb-664f-4e64-b0fb-3f29b220d4bc.png}{1}
\caption*{Foto 11: Git-Exportmeldung - siehe rechte untere Ecke}\label{F22841151}
\end{figure}


Sie können nun zu Git navigieren und sehen Ihre Dateien auf Git aufgelistet. Hiermit ist der Vorgang abgeschlossen.

\begin{figure}
\scaledgraphics{d80dc5a5-d385-4a3b-a03a-a752ff2686c9.png}{1}
\caption*{Foto 12: Ihre Publikation wird in Git ausgegeben}\label{F82558661}
\end{figure}


\subsection{Nächste Schritte}\label{H526381}



Sie können nun Ihr Team einladen, um auf die Publikation auf Fidus Writer zuzugreifen.

\end{document}

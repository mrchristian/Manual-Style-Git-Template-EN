\documentclass{article}

\begin{document}

\title{Über die Publikation}

\maketitle


Schnellstart-Anleitung - Eine Publishing-Pipeline


Vorabversion v0.1


Datum: 2022


Creative Commons Namensnennung-Weitergabe unter gleichen Bedingungen 4.0 International (CC BY-SA 4.0).


\subsection{Beschreibung}\label{H779765}



Bei der Pipeline geht es um die Automatisierung des Publizierens. Sie verbindet das Textverarbeitungsprogramm mit der direkten Veröffentlichung unter Verwendung von Single-Source-Publishing-Technologie und -Methoden.


\textbf{Single-Source-Publishing:} Bearbeiten Sie an einem Ort und verteilen Sie als Multi-Format automatisch an verschiedene Orte, um professionelle publikationsfertige Ausgaben zu produzieren.


Die Pipeline verbindet Fidus Writer, ein Online-Textverarbeitungsprogramm, mit GitLab oder GitHub zur versionierten Speicherung.


\subsection{Mitwirkende}\label{H8655734}



Autor(en): Simon Worthington - ORCID 0000-0002-8579-9717


\subsection{Technische Daten}\label{H3247814}



\subsubsection{Layout-Gestaltungsstil}\label{H8853078}



Der Publikations-Layoutstil 'Report 001' basiert auf der CSS-Vorlage von Interpunct - full stack graphic design, Interpunct.dev. GNU General Public License (GPLv3).


\subsubsection{Bilder}\label{H8994266}



Illustrationen Blush.design. Alle auf Blush veröffentlichten Illustrationen können frei verwendet werden. Lizenz https://blush.design/license.


Palette - https://coolors.co/ee4484-2062af-f68b1e-60bc55-000000


Körper: Schwarz und Orange


Rosa - EE4484


Blau - 2062AF


Orange - F68B1E


Grün - 60BC55


Schwarz - 000000


\subsubsection{Schriftarten}\label{H8602637}



Alle Schriftarten sind Open Licence Fonts.


Kopfzeilen - Fira Sans Condensed. Diese Schriftarten sind unter der Open Font License lizenziert. Dieses Projekt wird von Carrois, einer in Berlin ansässigen Schriftgießerei, geleitet. Um beizutragen, siehe github.com/mozilla/Fira.


Körper - Fira Sans. Diese Schriften sind unter der Open Font License lizenziert. Dieses Projekt wird von Carrois, einer Schriftgießerei mit Sitz in Berlin, geleitet. Um beizutragen, siehe github.com/mozilla/Fira


Logo - Source Sans Pro. Diese Schriften sind unter der Open Font License lizenziert. Source® Sans Pro, die erste Open-Source-Schriftfamilie von Adobe, wurde von Paul D. Hunt entworfen.


\subsubsection{Quelloffene Software}\label{H1183783}


\begin{itemize}
\item Fidus Writer: Akademische Online-Textverarbeitung - https://www.fiduswriter.org/


\item Vivliostyle: CSS-Satz - https://vivliostyle.org/


\item GitLab Gemeinschaftsausgabe: Git-Versionsverwaltungssystem - https://gitlab.com/rluna-gitlab/gitlab-ce


\item GitLab Pages: Veröffentlichung statischer Websites direkt aus einem Repository in GitLab - https://docs.gitlab.com/ee/user/project/pages/


\item Docsify: Website-Generator - https://docsify.js.org/\#/


\item Draw.io: Diagramm-Editor - https://github.com/jgraph/drawio


\item Inkscape: Vektorgrafik-Editor - https://inkscape.org/


\item GIMP: Bildbearbeitungsprogramm - https://www.gimp.org/


\item Scribus: Desk Top Publishing (DTP) - https://www.scribus.net/


\item Thoth: Verwaltung von Metadaten - https://thoth.pub/


\item OpenRefine: Datenbearbeitung - https://openrefine.org/


\item Wikidata: Verknüpfte offene Datenressourcen - https://www.wikidata.org/


\item Vereinheitlichtes medizinisches Sprachsystem (UMLS): Linked Open Data Schema für Medizin - https://www.nlm.nih.gov/research/umls/index.html


\item DSpace 7 und Dublin Core Metadata Initiative (DCMI) Metadatenschema: Standards für die Veröffentlichung von Metadaten - https://www.dublincore.org/specifications/dublin-core/dcmi-terms/


\item Wikimedia Commons: Repository - https://commons.wikimedia.org/


\item Ghostscript und Ghostmarks: PDF-Lesezeichen und Einfügen von Metadaten - Lesezeichen http://ask.xmodulo.com/add-bookmarks-pdf-document-linux.html und Metadaten https://milan.kupcevic.net/ghostscript-ps-pdf/


\item Zenodo: Forschungsrepositorium und DOI mint, persistenter Identifikator (PID) - https://zenodo.org/


\item ORCID: Forscher-Indentifikator, dauerhafter Identifikator (PID) - https://orcid.org/


\item ROR: Organisationsindentifikator, dauerhafter Identifikator (PID) - https://ror.org/


\item Zotero: Zitierverwaltung - https://www.zotero.org/


\item CrowdIn: Sprachübersetzung - https://crowdin.com/


\item DeepL: Sprachübersetzung - https://www.deepl.com/


\end{itemize}
\end{document}
